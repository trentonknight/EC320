\documentclass[pdflatex,12pt,a4paper]{report}
\usepackage[usenames,dvipsnames]{color}
\usepackage{tikz}
\usepackage{graphicx}
\usepackage{setspace}
\usepackage{hyperref}
\usepackage{url}
\usepackage{natbib}
%% Define a new 'leo' style for the package that will use a smaller font.
\makeatletter
\def\url@leostyle{%
  \@ifundefined{selectfont}{\def\UrlFont{\sf}}{\def\UrlFont{\small\ttfamily}}}
\makeatother
%% Now actually use the newly defined style.
\urlstyle{leo}
\newcommand{\HRule}{\rule{\linewidth}{0.5mm}}
\begin{document}
%
\begin{titlepage}
\begin{center}
% Upper part of the page
\textsc{\color{Sepia}{\LARGE EC~320}}\\[1.5cm]

\textsc{\Large Week-Eight}
\HRule\\[0.5cm]
{ \huge American Debt}\\[0.4cm]
\HRule 


% Author and supervisor
\begin{minipage}{0.4\textwidth}
\begin{flushleft} \large
\emph{Author:}\\
Jason \textsc{Mansfield}
\end{flushleft}
\end{minipage}
\begin{minipage}{0.4\textwidth}
\begin{flushright} \large
\emph{Instructor:} \\
Dr.~Anthony \textsc{Pizur}
\end{flushright}
\end{minipage}
\vfill

% Bottom of the page
{\large \today}

\end{center}
\end{titlepage}

\section{Debt and Deficit}
\begin{doublespace}
The United States debt has been spinning seemingly out of control. It appears that Congress is more concerned with competition and squabbling rather than concerning themselves with leading a country. Unemployment is rampant and the middle class is being driven into poverty. The White House Blog~\citep{phillips_infographic:_2011} has whats called an, \href{http://www.whitehouse.gov/infographics/us-national-debt}{"Infograph"} which details the government's mistakes made in the past which encouraged this financial erosion. In tune with this message the president almost sarcastically makes this statement when addressing the nation~\cite{curtis_colleen_president_2011}: 
\end{doublespace} 
\begin{citation}

The debate right now isn’t about whether we need to make tough choices.  Democrats and Republicans agree on the amount of deficit reduction we need. The debate is about how it should be done.  Most Americans, regardless of political party, don’t understand how we can ask a senior citizen to pay more for her Medicare before we ask a corporate jet owner or the oil companies to give up tax breaks that other companies don’t get.  How can we ask a student to pay more for college before we ask hedge fund managers to stop paying taxes at a lower rate than their secretaries?  How can we slash funding for education and clean energy before we ask people like me to give up tax breaks we don’t need and didn’t ask for?  
\end{citation}

\begin{doublespace}
Action speaks louder than words and it has become clear Congress is more concerned with lining their own pockets with gold rather than planning for the future generations. A New York Times Reporter makes this statement~\cite{tritch_how_2011}:  
\end{doublespace}
\begin{quotation}
A few lessons can be drawn from the numbers. First, the Bush tax cuts have had a huge damaging effect. If all of them expired as scheduled at the end of 2012, future deficits would be cut by about half, to sustainable levels. Second, a healthy budget requires a healthy economy; recessions wreak havoc by reducing tax revenue. Government has to spur demand and create jobs in a deep downturn, even though doing so worsens the deficit in the short run. Third, spending cuts alone will not close the gap. The chronic revenue shortfalls from serial tax cuts are simply too deep to fill with spending cuts alone. Taxes have to go up. 
\end{quotation}
\begin{doublespace}
So in agreement with Author Teresa Tritch, taxes must go up. It's inevitable that money must come from somewhere to deal with the debt created during the Bush years. Additionally what policies does the United States need to adopt to prevent any further corrosion of our dollar while paying these larger taxes?
\end{doublespace}
\section{Debt Intolerance}
\begin{doublespace}
The government need to form a fiscal policy which can safeguard the United States from making similar mistakes as we have made in the recent past. One approach is called Dept Intolerance: ~\cite{reinhart2003debt}
\end{doublespace}
\begin{quotation}
Debt intolerance is indeed intimately linked to the pervasive phenomenon of
serial default that has plagued so many countries over the past two centuries. Debt-intolerant
countries tend to have weak fiscal structures and weak financial systems. Default often
exacerbates these problems, making these same countries more prone to future default.
Understanding and measuring debt intolerance is fundamental to assessing the problems of debt
sustainability, debt restructuring, capital market integration, and to assessing the scope for
international lending to ameliorate crises.
\end{quotation}
\begin{doublespace}
Does this sound familiar? The United States is always in some form of debt in yet we continue to practice more spending than innovation. The aforementioned study uses mainly South American counties for examples but the comparison works. Granted during times of war we may need more resources then we can normally afford but the last few years should be proof positive we need to return to our American roots and begin innovation once again. While policies cannot in themselves create national products that increase our value, policies which encourage innovation and growth can. 
\end{doublespace}
\section{American Productivity and growth}
\begin{doublespace}
Productive industries and American innovation similar to the first Industrial Revolution. A McKinsey Quarterly author makes this statement~\cite{arthur2011second}:
\end{doublespace}
\begin{quotation}
In 1850, a decade before the Civil War, the United States’ economy
was small—it wasn’t much bigger than Italy’s. Forty years later, it was
the largest economy in the world. What happened in between was
the railroads. They linked the east of the country to the west, and the
interior to both. They gave access to the east’s industrial goods;
they made possible economies of scale; they stimulated steel and
manufacturing—and the economy was never the same.
\end{quotation}
\begin{doublespace}
This information is encouraging as it helps one to realize such growth is capable here in the United States of America. The United States has one massive resource which many countries do not have an abundance of education. A mixer of education and the willingness to be entrepreneurial could eventually save this country. Arena's such as biotechnology, energy and medical research are just some of the big players which could change not only the United States economy but the entire worlds.       
\end{doublespace}

\clearpage
% bib stuff
    \nocite{*}
    \bibliographystyle{apalike}
    \bibliography{cite}

\end{document}

